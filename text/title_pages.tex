% !TEX root = main.tex
\pagenumbering{roman}

\begin{titlepage}
\begin{center}
\ \\

\vspace{15mm}

\large
Charles University in Prague\\
Faculty of~Mathematics and Physics\\

\vspace{5mm}

{\Large\bf MASTER THESIS}

\vspace{15mm}

\includegraphics[scale=0.4]{logo.eps} %%% source http://www.mff.cuni.cz/fakulta/symboly/logo.eps

\vspace{20mm}
%\normalsize
{\Large Ondřej Plátek}\\ 

\vspace{5mm}
{\Large\bf Automatic speech recognition using Kaldi}

\vspace{20mm}
\large
\noindent
Institute of~Formal and Applied Linguistics\\
\noindent
Supervisor: Ing. Mgr. Filip Jurčíček, Ph.D.\\
\noindent
Study branch: Theoretical Computer Science\\
\end{center}
\vspace{20mm}
\begin{center}
Prague 2013
\end{center}

\end{titlepage} % zde koncí úvodní strana

%%%%%%%% 2-3 page with master thesis task
% FIXME -remove this from electronic version%%%%% 
\todon{Insert the~task $../images/zadani[1-2].png$ for the~printed version}
\newpage
%%%%%%%% end 2-3 page with master thesis task

%%% 4.(2.) title page
\normalsize % nastavení normální velikosti fontu
\vspace{10mm} 

\noindent I would like to thank my~supervisor, Ing. Mgr. Filip Jurčíček, Ph.D., for his advice, patience and guidence. I also thank to my~friends from Mff Lab, Ufal and Ada for keeping me in good spirits. Last but not least, I would like to thank my~parents for all the~help and support.


\vspace{\fill} % nastavuje dynamické umístění následujícího textu do spodní části stránky
\noindent
I declare that I wrote my master thesis independently and exclusively with~the~use of~the~cited sources. I agree with~lending and publishing this thesis.

%\medskip\noindent
%I acknowledge that my thesis is a~subject to the~stipulations of~rights and obligations of~the~Act No. 121/2000 Coll., Copyright Act as valid, especially the~fact that Charles University in~Prague has a right to conclude a~licence agreement on~the~use of~the~school work as per sect. 60, paragraph 1 of~the~Copyright Act.

\medskip\noindent
I declare that I carried out this master thesis independently, and only with the~cited sources, literature and other professional sources.

I understand that my work relates to the~rights and obligations under the~Act No. 121/2000 Coll., the~Copyright Act, as amended, in particular the~fact that the~Charles University in Prague has the~right to conclude a license agreement on the~use of~this work as a school work pursuant to Section 60 paragraph 1 of~the~Copyright Act.

\noindent Prague, August 1, 2013 \hspace{\fill}Ondřej Plátek 

%%%   Výtisk pak na tomto míst+ nezapomente PODEPSAT!
%%%                                         *********

%\begin{figure}[htp] \centering{
%\includegraphics[scale=0.82]{zadani1}}
%\end{figure}  
%
%\newpage
%\begin{figure}[htp] \centering{
%\includegraphics[scale=0.82]{zadani2}}
%\end{figure}  

\newpage

%%% Následuje strana s abstrakty. Doplnte vlastní údaje.
\vbox to 0.5\vsize{
\setlength\parindent{0mm}
\setlength\parskip{5mm}
Název práce: Rozpoznávání řeči pomocí Kaldi\\
Autor: Ondřej Plátek\\
Katedra: Ústav formální a aplikované lingvistiky\\
Vedoucí diplomové práce: Ing. Mgr. Filip Jurčíček, Ph.D.\\
E-mail vedoucího: jurcicek@ufal.mff.cuni.cz\\

\noindent Abstrakt:
\todo{Jednou z důležitých komponent v dialogovém systému je modul rozpoznávání mluvené řeči. Tématem této práce je  implementace výkonného rozpoznávače v open-source systému trénování ASR~Kaldi~(\href{http://kaldi.sourceforge.net/}{http://kaldi.sourceforge.net/}) pro dialogové systémy. 
Přestože KLADI již obsahuje ASR dekodéry, tak nejsou vhodné pro dialogové systémy z důvodu jejich malé optimalizace na rychlost a jejich velkého zpoždění v generování výsledku po ukončení promluvy. Proto hlavním cílem práce bude vyvinutí real-time rozpoznávače pro dialogové systémy optimalizovaného na rychlost a minimalizující zpoždění. Použité prostředky pro tuto optimalizaci mohou být například multi-vláknové dekódování nebo využití grafických karet pro obecné výpočty. Součástí této práce bude příprava akustického modelu a testování ve vyvíjeném dialogovém systému "Vystadial".}


\noindent Klíčová slova: ASR,rozpoznávání mluvené řeči, dekodér

\vspace{10mm}

\noindent
Title: Automatic speech recognition using Kaldi\\
Author: Ondřej Plátek\\
Department: Ústav formální a aplikované lingvistiky \\
Supervisor: Ing. Mgr. Filip Jurčíček, Ph.D.\\
Supervisor's e-mail address: jurcicek@ufal.mff.cuni.cz\\

\noindent Abstract: 
\todo{One of~the~important component of~a dialog system is a speech recognition module. The topic of~this thesis is to use speech recognition training system ASR~Kaldi~(\href{http://kaldi.sourceforge.net/}{http://kaldi.sourceforge.net/}) and implement efficient decoder for it. 
Kaldi is already deployed with decoders, but they are not convenient for dialog systems. The main goal of~this thesis is developing a real-time decoder for a dialog system, which minimize latency and optimize speed. Methods used for speeding up the~decoder are not limited to multi-threading decoding or usage of~GPU cards for general computations. Part of~this work is devoted to training of~an acoustic model and also testing it in the~"Vystadial" dialog system.
} % end todo

\noindent Keywords: ASR,speech recognition, decoder

\vss}

\newpage

\openright
\pagestyle{plain}
\pagenumbering{arabic}
\setcounter{page}{1}
