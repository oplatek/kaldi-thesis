% author: Ondrej Platek 2012
%settings are located in begin_settins.tex and end_settings.tex files
%do not remove!
% author Ondrej Platek 2012, based on Martin Mares settings
% \documentclass{beamer}
\documentclass[table]{beamer} % not work in beamer \usepackage[table]{xcolor}

%  %handout mode -both presentation and notes
%\documentclass[handout,notes=show]{beamer}
%\usepackage{pgfpages}
%\setbeameroption{show notes on second screen=left}
%\usetemplatenote{\beamertemplatefootempty \insertslideintonotes{0.7} \insertnote}

%  %only notes
%\documentclass[notes=only]{beamer}
%\usetemplatenote{\beamertemplatefootempty \insertslideintonotes{0.7} \insertnote}

    


%% If using csLaTeX (recommended):
%\usepackage{czech}
%% otherwise:
%\usepackage[czech]{babel}
%\usepackage[T1]{fontenc}

%% used encoding: obvykle latin2, cp1250 nebo utf8:
\usepackage[utf8x]{inputenc}

%% other packages 
\usepackage{graphicx}
\usepackage{amsthm}
\usepackage{multirow}  % for tables multirow

% Display lines for taking notes nex to the page
%\usepackage{handoutWithNotes} 
%\pgfpagesuselayout{1 on 1 with notes landscape}[a4paper,border shrink=5mm]

\beamersetuncovermixins{\opaqueness<1>{25}}{\opaqueness<2->{15}}

\def\todo#1{
\emph{\color{red} TODO: #1}
}

\def\todon#1{
  \todo{#1 \\}
}
%ruzna nastaveni beamru

%1)
%\setbeamertemplate{navigation symbols}{}
%\usetheme{Warsaw}
%2)
%\setbeamercolor{frametitle}{fg=black,bg=white}
%\setbeamercolor{title}{fg=black,bg=yellow!85!orange}
%\usetheme{AnnArbor}
%3)
%\usetheme{CambridgeUS}
%\usecolortheme{crane}
%4)
%\usetheme{Berkeley}
%\usecolortheme{beetle}

\usetheme{CambridgeUS}
\usecolortheme{beetle}
\setbeamercolor{title}{fg=black,bg=blue!20!green}
\setbeamercolor{frametitle}{fg=black,bg=blue!40!white}
\setbeamercolor{background canvas}{bg=white}


%
\author{Ondřej Plátek}
\title[Instance optimal query processing in SN]{Instance optimal query processing in spatial networks by Ke Deng, X. Zhou, Heng T. Shen, Shazzia S., Xue Li}  
% \date{\today}
\date{16.\,11.\,2010}
\subtitle{Seminar in Databases}
\institute[Mff]{Matematicko Fyzikální Fakulta Univerzity Karlovy}

\graphicspath{{../thesis-doc/images/}}

\begin{document}

../text/snippets/python_style.tex % do not remove

\maketitle

\section{Cíle} %%%%%%%%%%%%%%%%%%%%%%%%%%%%%%%%%%%%%%%%%%%%%%%%%%%%%%%%

\begin{frame}\frametitle{Cíle práce} 
    Zlepšit rozpoznávání řeči pro Alex Spoken Dialogue Systems \\
    Obzvlášt aplikaci poskytujcí informace o veřejné dopravě (800 899 998).
        \begin{exampleblock}{Zadané cíle práce:}
        \begin{itemize}
            \item přípravit akustické modely pomocí Kaldi toolkitu,
            \item vyvinout nový real-time rozpoznávač řeči, který podporuje inkrementální rozpoznávání řeči,
            \item integrovat rozpoznávač řeči do Alex SDS.
        \end{itemize}
    \end{exampleblock}
\end{frame}

\begin{frame}\frametitle{Kontinuální rozpoznávání řeči} 
    \begin{block}{Pattern matching}
        HMM --- modelování časové řady řeči (monofóny/triphóny pro slova)
        \begin{itemize}
            \item Natrénovali a porovnali jsme několik HMM akustických modelů.
        \end{itemize}
    \end{block}
    \begin{exampleblock}{Prohledávání grafu - dekódování}
        Viterbi algoritmus --- dynamické programování
        \begin{itemize}
            \item Změna rozhraní.
            \item Normalizace výstupu rozpoznávače.
            \item Nalézt optimální parametry (beam, lattice-beam, max-active-states).
        \end{itemize}
    \end{exampleblock}
\end{frame}

\section{Akustické modelování} %%%%%%%%%%%%%%%%%%%%%%%%%%%%%%%%%%%%%%%%%%%%%%%%%%%%%%%%%%%%%%%%%%

\begin{frame}\frametitle{Akustické modelování} 
    \input{images/asr-components-am}
    % Architecture of statistical speech recognizer\cite{ney1990acoustic}
\end{frame}

\begin{frame}\frametitle{ASR trénování, výsledky} 
    \begin{center}
    ../../text/images/am-deps.tex \\
    { \scriptsize
    \begin{tabular}{lr}
        \hline \\
        \theader{Czech} & \theader{bigram} \\
        tri $\Delta+\Delta\Delta$ &  \emph{{\color{red} {\tiny HTK} (60.4)} 56.6} \\
            tri LDA+MLLT &  53.9 \\
            tri LDA+MLLT+MMI &    49.5 \\
            tri LDA+MLLT+bMMI &   49.3 \\
            tri LDA+MLLT+MPE &    49.2 \\
        \hline \hline \\
        \theader{English} & \theader{bigram} \\ 
        tri $\Delta+\Delta\Delta$ & \emph{{\color{red} {\tiny HTK} (17.5)} 16.2}\\
            tri LDA+MLLT &    15.8 \\
            tri LDA+MLLT+MMI &    10.4  \\
            tri LDA+MLLT+bMMI &   10.2 \\
            tri LDA+MLLT+MPE &    11.1 \\
        \hline
    \end{tabular}
    }
    % \caption{Word error rates for zerogram and bigram LM for different training triphone methods.
    %     The~`tri~$\Delta+\Delta\Delta$' row shows results for a~generative model which is comparable to the~model trained using the~HTK scripts.
    % }
    \end{center}
\end{frame}


\section{On-line rozpoznávač} %%%%%%%%%%%%%%%%%%%%%%%%%%%%%%%%%%%%%%%%%%%%%%%%%%%%%%%%%%%%%%%%%%

\begin{frame}\frametitle{Komponenty on-line dekódování} 
    \begin{center}
        % \usepackage[usenames,dvipsnames]{pstricks}
% \usepackage{epsfig}
% \usepackage{pst-grad} % For gradients
% \usepackage{pst-plot} % For axes
% \psscalebox{1.0 1.0} % Change this value to rescale the drawing.
\scalebox{0.55}
{
\begin{pspicture}(0,-3.899762)(14.38,3.899762)
\definecolor{colour2}{rgb}{0.101960786,0.2627451,0.10980392}
\definecolor{colour0}{rgb}{0.13333334,0.41568628,0.7137255}
\definecolor{colour4}{rgb}{0.68235296,0.35686275,0.1764706}
\definecolor{colour3}{rgb}{0.27450982,0.3882353,0.50980395}
\psframe[linecolor=black, linewidth=0.04, dimen=outer](13.6,2.059762)(11.42,1.5197619)
\rput[bl](11.74,1.6597619){Decodable}
\psframe[linecolor=black, linewidth=0.04, dimen=outer](13.6,2.059762)(11.42,1.5197619)
\rput[bl](9.0,0.0){Decode(max\_frames)}
\psframe[linecolor=black, linewidth=0.04, dimen=outer](9.84,2.059762)(7.14,1.5197619)
\rput[bl](7.48,1.6397619){LDA+MLLT}
\psframe[linecolor=black, linewidth=0.04, dimen=outer](2.84,2.039762)(1.22,1.4997619)
\rput[bl](1.34,1.619762){Buffering}
\psframe[linecolor=black, linewidth=0.04, dimen=outer](5.82,2.059762)(4.2,1.5197619)
\rput[bl](4.58,1.6597619){Mfcc}
\psline[linecolor=black, linewidth=0.04, arrowsize=0.05291666666666667cm 2.0,arrowlength=1.4,arrowinset=0.1]{->}(3.04,1.7597619)(4.16,1.7597619)
\psline[linecolor=black, linewidth=0.04, arrowsize=0.05291666666666667cm 2.0,arrowlength=1.4,arrowinset=0.1]{->}(10.16,1.7597619)(11.28,1.7597619)
\psline[linecolor=black, linewidth=0.04, arrowsize=0.05291666666666667cm 2.0,arrowlength=1.4,arrowinset=0.1]{->}(5.96,1.799762)(7.08,1.799762)
\psframe[linecolor=black, linewidth=0.04, dimen=outer, doubleline=true, doublesep=0.12](8.58,0.15976197)(5.7,-1.200238)
\rput[bl](6.56,-0.64023805){Decoder}
\psline[linecolor=black, linewidth=0.04, arrowsize=0.05291666666666667cm 2.0,arrowlength=1.4,arrowinset=0.1]{->}(11.22,1.4997619)(7.2,0.77976197)(7.224516,0.19976196)
\rput[bl](3.18,2.059762){$50ms$}
\rput[bl](6.14,2.179762){$a_i$}
\rput[bl](10.22,2.079762){$a_i^t$}
\rput[bl](6.78,0.699762){$l_i$}
\psellipse[linecolor=black, linewidth=0.04, dimen=outer](7.19,1.119762)(7.19,2.78)
\psframe[linecolor=black, linewidth=0.04, dimen=outer](5.7,-2.840238)(3.1,-3.600238)
\rput[bl](3.46,-3.2802382){GetLattice}
\psframe[linecolor=black, linewidth=0.04, dimen=outer](12.52,-2.840238)(6.98,-3.600238)
\rput[bl](7.38,-3.420238){CompactLatticeToWordPost}
\psline[linecolor=black, linewidth=0.04, doubleline=true, doublesep=0.12, arrowsize=0.05291666666666667cm 2.0,arrowlength=1.4,arrowinset=0.1]{->}(6.8,-1.320238)(4.28,-2.820238)
\psline[linecolor=black, linewidth=0.04, doubleline=true, doublesep=0.12, arrowsize=0.05291666666666667cm 2.0,arrowlength=1.4,arrowinset=0.1]{->}(5.78,-3.220238)(6.98,-3.200238)
\rput[bl](5.14,3.299762){\textcolor{colour2}{Forward decoding: every 10ms}}
\rput[bl](5.48,-2.5402381){\textcolor{colour0}{Backward decoding: Once per uterance}}
\psline[linecolor=colour3, linewidth=0.04, linestyle=dashed, dash=0.17638889cm 0.10583334cm, shadow=true,shadowsize=0.10583334,shadowcolor=colour4, arrowsize=0.05291666666666667cm 2.0,arrowlength=1.4,arrowinset=0.1]{->}(7.16,-3.580238)(6.32,-3.880238)(5.42,-3.680238)
\psline[linecolor=colour3, linewidth=0.04, linestyle=dashed, dash=0.17638889cm 0.10583334cm, shadow=true,shadowsize=0.10583334,shadowcolor=colour4, arrowsize=0.05291666666666667cm 2.0,arrowlength=1.4,arrowinset=0.1]{->}(8.58,-0.48023805)(12.38,0.37976196)(12.4,1.3997619)
\psline[linecolor=colour3, linewidth=0.04, linestyle=dashed, dash=0.17638889cm 0.10583334cm, shadow=true,shadowsize=0.10583334,shadowcolor=colour4, arrowsize=0.05291666666666667cm 2.0,arrowlength=1.4,arrowinset=0.1]{->}(12.26,2.079762)(11.24,2.819762)(9.68,2.819762)(9.16,2.279762)
\psline[linecolor=colour3, linewidth=0.04, linestyle=dashed, dash=0.17638889cm 0.10583334cm, shadow=true,shadowsize=0.10583334,shadowcolor=colour4, arrowsize=0.05291666666666667cm 2.0,arrowlength=1.4,arrowinset=0.1]{->}(7.98,2.159762)(7.28,2.9397619)(5.82,2.9597619)(4.9,2.139762)
\psline[linecolor=colour3, linewidth=0.04, linestyle=dashed, dash=0.17638889cm 0.10583334cm, shadow=true,shadowsize=0.10583334,shadowcolor=colour4, arrowsize=0.05291666666666667cm 2.0,arrowlength=1.4,arrowinset=0.1]{->}(4.54,2.119762)(3.92,2.859762)(2.74,2.879762)(2.02,2.1997619)
\end{pspicture}
}

    \end{center}
\end{frame}

\begin{frame}\frametitle{(Py)OnlineLatgenRecogniser rozhraní} 
    \begin{itemize}
        \item \term{AudioIn} -- zařazení audio do fronty k předzpracování
        \item \term{Decode} -- dekódování určitý počet audio rámců (frame)
        \item \term{PruneFinal} -- příprava datových struktur pro extrakci lattice (svazu)
        \item \term{GetLattice} -- extrakce slovní posteriorní lattice
        \item \term{GetBestPath} -- extrakce nejpravděpodobnější slovní hypotézy
        \item \term{Reset} -- příprava rozpoznávače na novou promluvu
    \end{itemize}
\end{frame}


\begin{frame}\frametitle{Evaluace, metriky} 
    \begin{itemize}
        \item Real Time Factor (RTF) dekódování -- poměr času dekódování a délky promluvy,
        \item Latence -- zpozdění mezi koncem promluvy a dostupností výsledků rozpoznávání,
        \item Word Error Rate (WER) -- chyba nejlepší slovní transkripce.
    \end{itemize}
\end{frame}

\section[Evaluace na "PTI"]{Evaluace na "Public Transport Information" doméně}%%%%%%%%%%%%%%%%%%%%%%%%%%%%

\begin{frame}\frametitle{On-line vs dávkové dekódování} 
    \begin{center}
        \includegraphics[scale=0.5]{lat_cloud_kaldi.png}
    \end{center}
    \bf {Pokles WER 45 \% $\longrightarrow$ 22\%} pro náš dialogový systém Alex
    % \caption{Almost constant latency of on-line decoder (OnlineLatgenRecogniser) and linearly growing latency of cloud based speech recogniser (Google ASR service) for increasing utterance length.}
\end{frame}

\begin{frame}\frametitle{Public Transport Information doména - rychlost a přesnost} 
    \includegraphics[scale=0.35]{beam_vs_rtfwer.pdf.ps}
    \includegraphics[scale=0.35]{latbeam_vs_latwer.pdf.ps}
    \\  Dostatečně rychlé -- pod 200 ms -- pro 95 \% promluv. \\
    \includegraphics[scale=0.35]{frtf_vs_prc.pdf.ps}
    \includegraphics[scale=0.35]{lat_vs_prc.pdf.ps}
    % \caption{The~upper graph (a) shows that WER decreases with increasing \term{beam} and the~average RTF linearly grows with the~beam.
    %     The~growth of the~95th RTF percentile is limited at 0.6 by setting \term{max-active-states} to 2000, because the~\term{max-active-states} parameters influence presumably the~worst cases with large search space.
    % The~lower graph (b) shows latency growth in response to increasing \term{lattice-beam}.}
\end{frame}


\section{Shrnutí} %%%%%%%%%%%%%%%%%%%%%%%%%%%%%%%%%%%%%%%%%%%%%%%%%%%%%%%%%%%%%%%%%%

% \begin{frame}\frametitle{Výsledky} 
%     \begin{itemize}
%         \item WER 22, latence pod 200 ms na české PTI doméně (dříve 1900 ms a 48 WER). 
%         \item WER 50 pro češtinu na Vystadial datesetu (čeština a komplexní doména)
%         \item WER 12 pro angličtinu na Vystadial datasetu 
%     \end{itemize}
% \end{frame}

\begin{frame}\frametitle{Shrnutí} 
    \begin{exampleblock}{Výsledky}
    \begin{itemize}
        \item V dialogovém systému WER 22, latence pod 200 ms. \\ {\scriptsize Dříve 1900 ms a 48 WER}. 
        \item WER pro Vystadial skripty: angličtina 12, čeština mix domén 50
    \end{itemize}
    \end{exampleblock}
    \begin{block}{Závěry}
    \begin{itemize}
        \item Testovaný real-time on-line rozpoznávač řeči
        \item Trénovací skripty pro češtinu a angličtinu {\scriptsize (Vystadial)}-- přijmuto do Kaldi
        \item Integrace rozpoznávače řeči do dialogového systému, v reálném provozu na lince 800 899 998 (PTI doména)
        \item Spoluautor akceptovaných článků na konference Sigdial, Lrec a TSD (Viz reference) 
    \end{itemize}
    \end{block}
\end{frame}


\section{Podpůrné slidy} %%%%%%%%%%%%%%%%%%%%%%%%%%%%%%%

\begin{frame}\frametitle{Přesnost akustických modelů dle velikosti dat} 
    \begin{center}
        \includegraphics[scale=0.7]{images/partial-zerogram.ps}
        % \caption{The~figure displays improving performance of Czech generative AM based on growing size of training data for acoustic modelling. The~zerogram LM allows to evaluate only acoustic modelling, but causes a~high WER. }
    \end{center}
\end{frame}

\begin{frame}\frametitle{Přesnost akustických modelů dle velikosti trénovací dat LM} 
    \begin{center}
        \includegraphics[scale=0.7]{images/partial-lm-tri2b-bmmi.ps}
        % \caption{Influence of in-domain text size of \ac{LM} on speech recognition quality. The~\ac{AM} \term{tri2b\_bmmi} and parameters are fixed and only \ac{LM} training size varies.}
    \end{center}
\end{frame}


\begin{frame}\frametitle{Vystadial dataset} 
    \begin{center}
    Posbíráno skupinou UFAL Dialogovým systémů.\\
    \begin{tabular}{lrrr}
        \hline
        dataset & audio[h] & \# vět & \# slov \\
        \hline
        \textbf{English} & & & \\
                training & 41:30 & 47,463 & 178,110 \\
                development & 01:45 & 2,000 & 7,376 \\
                test & 01:46 & 2,000 & 7,772 \\
        \hline
        \textbf{Czech} & & & \\
                training & 15:25 & 22,567 & 126,333 \\
                development & 01:23 & 2,000 & 11,478 \\
                test & 01:22 & 2,000 & 11,204 \\
        \hline
		\end{tabular}
    % \caption{Size of the~data: length of the~audio (hours:minutes), number of sentences
        % (which is the~same as the~number of recordings), number of words in the~
    % transcriptions.\cite{korvas_2014}}
    \end{center}
\end{frame}


\begin{frame}\frametitle{Acoustic features, features preprocessing} 
    ../../text/images/mfcc_window.tex
    ../../text/images/mfcc-delta.tex
    % \caption{\ac{PLP} or \ac{MFCC} features are computed every 10 ms seconds in 25 ms windows.
    % Audio length is $(frames-1)*shift + win\_len = 85ms$}
    % \caption{Typical setup with 39 features using \ac{MFCC}.}
\end{frame}

\begin{frame}[fragile]\frametitle{Výstupní formáty} 

    \begin{verbatim}
    0.5 hi how are you
    0.2 hi where are you
    0.1 bey how are you
\end{verbatim}

% Example of 3-best list output with posterior probability for each path. 
% N-best list in Kaldi can be easily extracted from lattices. 
    \begin{center}
        \includegraphics[width=30em]{toy_lattice.ps}
    \end{center}
    % \caption{Word posterior lattice. 
    %     Common parts of hypotheses are effectively represented. 
    %     All outgoing arcs for each node sum to 1.0. }
\end{frame}


\begin{frame}\frametitle{Funkční (Py)OnlineLatgenRecogniser demo} 
    \begin{center}
        \lstinputlisting[style=Python]{pykaldi_usage.py}
    \end{center}
\end{frame}

\begin{frame}\frametitle{Problém} 
    Dialogové systémy potřebují rozpoznávání řeči \\
    OpenJulius --- padá, RWTH decoder --- licence \\
    Cloudové služby Google a Nuance --- žádná adaptace + problémy s licencemi 
\end{frame}

\begin{frame} \frametitle{Semiring}
\begin{tabular}{lrrrrr}
\hline
Name & $\mathcal{K}$ & $\oplus$ & $ \otimes$ & $\bar{0}$ & $\bar{1}$ \\ 
\hline
Real        & $[0,\infty)$        &  +                     &  * &  0        &  1  \\
Log         & $(-\infty, \infty)$ & $-log(e^{-x} + e^{-y})$ & + &  $\infty$ &  0  \\
Tropical    & $(-\infty, \infty)$ &  min                   &  + &  $\infty$ &  0  \\
\hline
\end{tabular}
% \caption{Semirings used in speech recognition.\cite{openfst_web}}
\end{frame}

\begin{frame}\frametitle{Odkazy a reference}
    Děkuji za pozornost! \\
    \begin{exampleblock}{Související odkazy}
        \begin{itemize}
            \item Diplomová práce \\ {\bf \url{https://github.com/oplatek/kaldi-thesis}}
            \item OnlineLatgenRecogniser implementace a AM trénovací skripty \\ {\bf \url{https://github.com/UFAL-DSG/pykaldi}} 
            \item Alex implementace \\ {\bf \url{https://github.com/UFAL-DSG/alex}} 
            % \item Alex web presentation \\ {\bf \url{http://ufal.mff.cuni.cz/alex-dialogue-systems-framework/}} 
            % \item Kontakt \& CV \\ {\bf \url{http://www.linkedin.com/in/ondrejplatek}}
        \end{itemize}
    \end{exampleblock}
    \begin{exampleblock}{Reference}
        \tiny
        \begin{itemize}
            \item Vystadial dataset --  Matěj Korvas, Ondřej Plátek, Ondřej Dušek, Lukáš Žilka, and Filip Jurčíček, Free English and Czech telephone speech corpus shared under the CC-BY-SA 3.0 license, Proceedings of the Eight International Conference on Language Resources and Evaluation (LREC 2014), 2014, p. To Appear.
            \item Free on-line speech recogniser based on Kaldi ASR toolkit producing word posterior lattices -- Ondřej Plátek and Filip Jurčíček, Proceedings of the Sigdial 2014 conference
            \item Integration of an online Kaldi speech recogniser to Alex Dialogue Systems Framework -- Ondřej Plátek and Filip Jurčíček, TSD conference 2014
        \end{itemize}
    \end{exampleblock}
\end{frame}

\end{document}  % do not remove!
